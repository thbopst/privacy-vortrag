\section{E-Mail-Verschlüsselung}

\begin{frame}
  \frametitle{Probleme symmetrischer Verschlüsselung}
  \begin{itemize}
    \item Verteilung der Schlüssel
    \item Anzahl benötigter Schlüssel
    \begin{itemize}
      \item 2 Personen $\rightarrow$ 1 Schlüssel
      \item 3 Personen $\rightarrow$ 3 Schlüssel
      \item 4 Personen $\rightarrow$ 6 Schlüssel
      \item 100 Personen $\rightarrow$ 4950 Schlüssel
      \item n Personen $\rightarrow$ $\frac{n(n-1)}{2}$ Schlüssel
    \end{itemize}
    \item TODO: Graf
  \end{itemize}

\end{frame}

\begin{frame}
   \frametitle{Asymmetrische Verschlüsselung} 
   \begin{itemize}
     \item Schaubild
   \end{itemize}
\end{frame}

\begin{frame}
  \frametitle{Keyserver}
  \begin{itemize}
    \item Screenshots
  \end{itemize}
\end{frame}

\begin{frame}
  \frametitle{Identitätsprüfung}
  \begin{itemize}
    \item Keine Echtheitsgarantie für Keys auf Keyserver
    \item ``Web of Trust''
    \item TODO: Schaubild (ger. graph)
    \note{Key nur signieren wenn WIRKLICH geprüft}
  \end{itemize}
\end{frame}

\begin{frame}
  \frametitle{Signaturen}
  \begin{itemize}
    \item Schaubild Signaturschema
    \item Schaubild Signatur generieren + prüfen mit pub/sec key
  \end{itemize}
\end{frame}

\begin{frame}
  \frametitle{Signierte E-Mail}
  \begin{itemize}
    \item Garantiert Echtheit des Absenders
    \item Keine Verschlüsselung!
    \note{Unterschriebener Vertrag}
    \item Kann aber mit Verschlüsselung kombiniert werden
  \end{itemize}
\end{frame}